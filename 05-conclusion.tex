\section{Conclusion}

The recent advancements of machine learning algorithms make them ideal
candidates to solve complex problems. However, using these solutions in order
to process problems at scale is still a daunting task. In particular, knowledge
of parallel and distributed computing techniques is necessary to facilitate the
development of such systems and streamline their execution time.

In this paper, we shared our experience in developing a highly scalable and
efficient solution for a stochastic gradient Markov chain Monte Carlo algorithm
that detects overlapping communities in graphs. The system design had to
overcome several obstacles in order to achieve high performance. Specifically,
we discussed how the algorithm was structured to facilitate its
parallelization. Moreover, we evaluated the efficacy of overlapping computation
with communication to hide latency.  Further, we demonstrated the use of a
mixture of MPI and RDMA primitives to speedup the communication between cluster
nodes.

We conducted a thorough empirical evaluation of the system to study its strong
and weak scalability on 65 cluster nodes using large data sets.  Additionally,
we assessed the efficiency of the algorithm's resource utilization. Finally, a
demonstration of the implementation's utility was provided by processing 6
different organic data sets.
